
\section{Other Linux tools}
\subsection{checksums}
Sometimes there is a need to calculate check sum of the file.
Calculating check sum is nothing more than taking a binary file and using the algorithm calculating short string that fully depends on content of the file.
In other words if only one letter or space has been changed in file check sum will be looking totally different.
There are many algorithms like MD5, SHA1 or SHA2 that we can use.
In order to calculate checksum on Linux type in terminal:
\texttt{md5sum <file-name>} or \texttt{sha1sum <file-name>} etc.
%%%%%%%%%%%%%%%%%%%%%%%%%%%%%%%%%%%%%%%%%%%%%%%%%%%%%%%%%%%%%%%%%%%%%%%%%%%%%%%%%%%%%%%
\subsection{wget}
\subsubsection{wget ftp://}
In order to download all ftp files from ftp server you must use wget as: \newline \newline
\q{wget -m ftp://<login><password>@<host\_ip\_address>} \newline \newline
Where \q{-m} stands for \q{--mirror} and allows to recursively copy all the home directory from remote ftp host .Another options you can use here are:
\begin{itemize}
    \item \q{--no-passive} -> if your PASV transfer fails.
\end{itemize}
%%%%%%%%%%%%%%%%%%%%%%%%%%%%%%%%%%%%%%%%%%%%%%%%%%%%%%%%%%%%%%%%%%%%%%%%%%%%%%%%%%%%%%%%%%%%%%
\subsection{strings}

The strings command returns each string of printable characters in files.
The main usage is to determine the contents of and to extract text from binary files (i.e., non-text files).
Example syntax could be: \newline
\q{strings -n 8 backup.mdb >> wordlist.txt}\newline
so it will produce the wordlist text file with printable strings with at least 8 characters.
%%%%%%%%%%%%%%%%%%%%%%%%%%%%%%%%%%%%%%%%%%%%%%%%%%%%%%%%%%%%%%%%%%%%%%%%%%%%%%%%%%%%%%%%%%%%%%
\subsubsection{john}
\label{subsubsec:john}
The John tool allows to crack zip archives.
It uses brute force method so you must provide wordlist for that.
Example syntax is: \newline
\q{john }
%%%%%%%%%%%%%%%%%%%%%%%%%%%%%%%%%%%%%%%%%%%%%%%%%%%%%%%%%%%%%%%%%%%%%%%%%%%%%%%%%%%%%%%%%%%%%
\subsubsection{mdbtools}
%%%%%%%%%%%%%%%%%%%%%%%%%%%%%%%%%%%%%%%%%%%%%%%%%%%%%%%%%%%%%%%%%%%%%%%%%%%%%%%%%%%%%%%%%%%%%%
\subsubsection{zip2john}
zip2john utility is used to get hashed password out of zip archive.
rar2john will work the same but for rar archives.
Example usage is:\newline
\q{zip2john encrypted.zip > encrypted.hash} \newline
Afterwards to see the password you can use john the ripper - \q{john} \ref{subsubsec:john} as follows: \newline
\q{john  --show encrypted.hash}
%%%%%%%%%%%%%%%%%%%%%%%%%%%%%%%%%%%%%%%%%%%%%%%%%%%%%%%%%%%%%%%%%%%%%%%%%%%%%%%%%%%%%%%%%%%%%%
\subsubsection{file}
\label{subsubsec:file}
This tools shows the headers of the file and determines the format of the file.
The output might be corrupted since it only checks file headers that could be intentionally overwritten (see \ref{subsec:phpinjection} for example).
%%%%%%%%%%%%%%%%%%%%%%%%%%%%%%%%%%%%%%%%%%%%%%%%%%%%%%%%%%%%%%%%%%%%%%%%%%%%%%%%%%%%%%%%%%%%%%
\subsubsection{exiftool}
\label{subsubsec:exiftool}
This tool prints header information about file.
It is similar to file but gives more datailed description like encryption method, number of bytes etc etc.
%%%%%%%%%%%%%%%%%%%%%%%%%%%%%%%%%%%%%%%%%%%%%%%%%%%%%%%%%%%%%%%%%%%%%%%%%%%%%%%%%%%%%%%%%%%%%%

%%%%%%%%%%%%%%%%%%%%%%%%%%%%%%%%%%%%%%%%%%%%%%%%%%%%%%%%%%%%%%%%%%%%%%%%%%%%%%%%%%%%%%%%%%%%%%

%%%%%%%%%%%%%%%%%%%%%%%%%%%%%%%%%%%%%%%%%%%%%%%%%%%%%%%%%%%%%%%%%%%%%%%%%%%%%%%%%%%%%%%%%%%%%%

%%%%%%%%%%%%%%%%%%%%%%%%%%%%%%%%%%%%%%%%%%%%%%%%%%%%%%%%%%%%%%%%%%%%%%%%%%%%%%%%%%%%%%%%%%%%%%

%%%%%%%%%%%%%%%%%%%%%%%%%%%%%%%%%%%%%%%%%%%%%%%%%%%%%%%%%%%%%%%%%%%%%%%%%%%%%%%%%%%%%%%%%%%%%%
\section{Appendix}

\subsection{Linux Bonus}
\paragraph{passwd} This text file consist of users and passwords information.
It consist of bunch of lines like: \newline
\texttt{postgres:x:121:130:PostgreSQL administrator,,,:/var/lib/postgresql:/bin/bash} , which means:
\texttt{user-name:(x)encrypted-password?:id:group-id:system-name,,,:home-folder:default-shell} \newline
Ids larger than 99 are created after system instalation, ids between 0-99 are reserved for "users" created during system installation.
Passwords encryption file is called \texttt{shadow} and also located in \texttt{../etc} directory.

%%%%%%%%%%%%%%%%%%%%%%%%%%%%%%%%%%%%%%%%%%%%%%%%%%%%%%%%%%%%%%%%%%%%%%%%%%%%%%%%%%%%%%%

\subsection{SSH Tunnel - Local and Remote Port Forwarding}
%todo add more details
Port forwarding is the act of configuring a router to make a computer or network device (that is connected to) accessible to other computers and network devices outside the local network.
Port forwarding is all about setting proxy and forwarding connections via ssh.
It can be used for adding encryption to legacy applications, going through firewalls, and some system administrators and IT professionals use it for opening backdoors into the internal network from their home machines.
If you do not have public IP or your IP is blocked on some site you can use proxy forwarding.
In the such situation you must ensure that another machine (or localhost) has ssh configured.
This machine can be any computer connected to the internet.
In order to configure ssh on your machine(s) (assuming you are using OpenSSH server) you must open \q{/etc/ssh/sshd\_config} and set \q{AllowTcpForwarding=Yes}.
Additionally if you are using remote port forwarding you must set \q{GatewayPorts=Yes}.
After that restart ssh client by: \q{sudo service sshd restart}.

%todo finish that
\subsubsection{Local port forwarding}
Local port forwarding allows you to forward traffic on a port of your local computer to the SSH server, which is forwarded to a destination server.
In other words connection via ssh is redirected by your machine to the another server.


\texttt{ssh -L <local\_port\_to\_forward\_from>:<remote\_host\_A>:<destination\_port\_from\_remote\_host> <login>@<local\_host>} \newline\newline
Here ssh establishes connection and listens on \q{local\_port\_to\_forward\_from} and redirects the traffic to \q{destination\_port\_from\_remote\_host}.
Now if you send a request (or point the address into browser) to \q{http://localhost:<local\_port\_to\_forward\_from>} you will be redirected to \q{<host\_to\_forward\_to>}

\paragraph{Example} Forwarding local port to \q{http://internet.com:80}. \newline \newline
\texttt{ssh -L 9001:www.internet.com:80 localhost} \newline \newline
Here you can instead of "localhost" type your local ip address od name of your computer.
Now if you go to browser and type \q{localhost:9001} you will be redirected to \q{www.internet.com:80}.
Notice that traffic goes only through your own network card.
\newline
\newline
Let's have a look at another scenario.
For instance if you have your own computer A and "mistification" user on  computer B (\q{10.10.10.1}) somewhere (or virtual machine) that is trusted to some other server C (\q{123.456.789.000}) you can forward traffic from ports A to B and to remote server C in order to redirect it to back to B and A again.
Look at the example: \newline \newline
\texttt{ssh -L 9004:123.456.789.000:80 mistification@10.10.10.1} \newline \newline
This will establish connection between your machine and \q{10.10.10.1} and will forward the traffic coming from your machine from 9004 port to ssh server on \q{10.10.10.1}.
Now you established ssh tunnel between your machine, remote machine and server.
If you type in your browser \q{localhost:9004} you will be redirected to \q{123.456.789.000:80} (it can be any site instead of IP like www.internet.com or else).
But now traffic goes also through \q{10.10.10.1} network card.
You can easily check it by running wireshark (see \ref{subsec:wireshark}) on \q{10.10.10.1} and filtering by "ssh".
Now you must call the \q{localhost:9004} and you will see packets going through \q{10.10.10.1} network interface.
Always if you have exchange key saved for \q{10.10.10.1} you can use ssh forwarding it as:\newline \newline

\q{ssh -i <path\_to\_key> -L 9004:123.456.789.000:80 10.10.10.1}

\subsubsection{Remote port forwarding}

Remote port forwarding it the opposite thing to local in terms of traffic forwarding.
When you set up the connection and remote host calls some site or ip address the traffic goes through your destination server and port.
Look at the syntax example: \newline\newline


\q{ssh -R 9004:127.0.0.1:8080 admin@mistificated.global}\newline\newline

In this case ssh establishes connection between your machine and binds to port \q{9004} on \q{mistificated.global}.
Now if someone calls the \q{https://mistificated.global:9004} (and it does not matter whe the calling machine is from) it will see traffic that is to be transmitted on \q{127.0.0.1:8080 on your machine}.
So if you have an application working on \q{localhost:8080/WebGoat} (for instance WebGoat jsp app deployed locally on tomcat) and you will set up ssh port forwarding by: \newline
\q{ssh -R 9004:localhost:8080 admin@mistificated.global} \newline
and if user hits \q{https://mistificated.blobal:9004} in his browser, user will se the WebGoat.jsp application.

%%%%%%%%%%%%%%%%%%%%%%%%%%%%%%%%%%%%%%%%%%%%%%%%%%%%%%%%%%%%%%%%%%%%%%%%%%%%%%%%%%%%%%%
%%%%%%%%%%%%%%%%%%%%%%%%%%%%%%%%%%%%%%%%%%%%%%%%%%%%%%%%%%%%%%%%%%%%%%%%%%%%%%%%%%%%%%
\subsection{Ciphering Protocols}
%%%%%%%%%%%%%%%%%%%%%%%%%%%%%%%%%%%%%%%%%%%%%%%%%%%%%%%%%%%%%%%%%%%%%%%%%%%%%%%%%%%%%%%

\subsection{Important sites}
\texttt{www.netcraft.com}, provides information about underlying software
\texttt{www.securityfocus.com} vulnerabilities
\texttt{www.packetstormsecurity} vulnerabilities
\texttt{www.exploit-db.org} vulnerabilities
\texttt{www.cve.mitre.org} vulnerabilities
\texttt{www.osvdb.com} vulnerabilities repository
\texttt{www.owasp.org} the free and open software security community
\texttt{www.hackthebox.com} provides dozens of virtualmachines that you can practise on.
\texttt{}
\texttt{}
\texttt{}


%%%%%%%%%%%%%%%%%%%%%%%%%%%%%%%%%%%%%%%%%%%%%%%%%%%%%%%%%%%%%%%%%%%%%%%%%%%%%%%%%%%%%%%
\subsection{TLS handshake}
%%%%%%%%%%%%%%%%%%%%%%%%%%%%%%%%%%%%%%%%%%%%%%%%%%%%%%%%%%%%%%%%%%%%%%%%%%%%%%%%%%%%%%%

\subsection{Regexp, grep and regular expressions}
\subsection{Bash scripts}
\subsubsection{Curl}
\label{subsubsec:curl}




%%%%%%%%%%%%%%%%%%%%%%%%%%%%%%%%%%%%%%%%%%%%%%%%%%%%%%%%%%%%%%%%%%%%%%%%%%%%%%%%%%%%%%%
\subsection{OSI - Open Systems Interconnection \cite{osil} }
\label{subsec:osi}
The Open System Interconnection (OSI) model defines a networking framework to implement protocols in seven layers.
There is really nothing to the OSI model.
In fact, it's not even tangible.
The OSI model doesn't perform any functions in the networking process.
It is a conceptual framework so we can better understand complex interactions that are happening.


\paragraph{Application (Layer 7)}

OSI Model, Layer 7, supports application and end-user processes.
Communication partners are identified, quality of service is identified, user authentication and privacy are considered, and any constraints on data syntax are identified.
Everything at this layer is application-specific.
This layer provides application services for file transfers, e-mail, and other network software services.
Telnet and FTP are applications that exist entirely in the application level.
Tiered application architectures are part of this layer.

\subparagraph{examples} WWW browsers, NFS, SNMP, Telnet, HTTP, FTP


\paragraph{Presentation (Layer 6)}
This layer provides independence from differences in data representation (e.g., encryption) by translating from application to network format, and vice versa.
The presentation layer works to transform data into the form that the application layer can accept.
This layer formats and encrypts data to be sent across a network, providing freedom from compatibility problems.
It is sometimes called the syntax layer.

\subparagraph{examples} encryption, ASCII, EBCDIC, TIFF, GIF, PICT, JPEG, MPEG, MIDI.



\paragraph{Session (Layer 5)}
This layer establishes, manages and terminates connections between applications.
The session layer sets up, coordinates, and terminates conversations, exchanges, and dialogues between the applications at each end.
It deals with session and connection coordination.

\subparagraph{examples} NFS, NetBios names, RPC, SQL.



\paragraph{Transport (Layer 4)}
OSI Model, Layer 4, provides transparent transfer of data between end systems, or hosts, and is responsible for end-to-end error recovery and flow control.
It ensures complete data transfer.
\subparagraph{examples} SPX, TCP, UDP.


\paragraph{Network (Layer 3)}

Layer 3 provides switching and routing technologies, creating logical paths, known as virtual circuits, for transmitting data from node to node.
Routing and forwarding are functions of this layer, as well as addressing, internetworking, error handling, congestion control and packet sequencing

\subparagraph{examples} AppleTalk DDP, IP, IPX.



\paragraph{Data Link (Layer 2)}
At OSI Model, Layer 2, data packets are encoded and decoded into bits.
It furnishes transmission protocol knowledge and management and handles errors in the physical layer, flow control and frame synchronization.
The data link layer is divided into two sub layers: The Media Access Control (MAC) layer and the Logical Link Control (LLC) layer.
The MAC sub layer controls how a computer on the network gains access to the data and permission to transmit it.
The LLC layer controls frame synchronization, flow control and error checking.
\subparagraph{examples} PPP, FDDI, ATM, IEEE 802.5/ 802.2, IEEE 802.3/802.2, HDLC, Frame Relay.


\paragraph{Physical (Layer 1)}
OSI Model, Layer 1 conveys the bit stream - electrical impulse, light or radio signal - through the network at the electrical and mechanical level.
It provides the hardware means of sending and receiving data on a carrier, including defining cables, cards and physical aspects.
Fast Ethernet, RS232, and ATM are protocols with physical layer components.
\subparagraph{examples} Ethernet, FDDI, B8ZS, V.35, V.24, RJ45.

\subsection{Cross-Orgin Resource Sharing}
Cross-Origin Resource Sharing (CORS) is a mechanism that uses additional HTTP headers to tell a browser to let a web application running at one origin (domain) have permission to access selected resources from a server at a different origin.
A web application executes a cross-origin HTTP request when it requests a resource that has a different origin (domain, protocol, and port) than its own origin.
For security reasons, browsers restrict cross-origin HTTP requests initiated from within scripts.
For example, XMLHttpRequest and the Fetch API follow the same-origin policy.
This means that a web application using those APIs can only request HTTP resources from the same origin the application was loaded from, unless the response from the other origin includes the right CORS headers.
%%%%%%%%%%%%%%%%%%%%%%%%%%%%%%%%%%%%%%%%%%%%%%%%%%%%%%%%%%%%%%%%%%%%%%%%%%%%%%%%%%%%%%
\subsection{Linux Priviledge Escalation}

g0tmi1k: (Linux) privilege escalation is all about:
\newline
Collect - Enumeration, more enumeration and some more enumeration. \newline
Process - Sort through data, analyse and prioritisation. \newline
Search - Know what to search for and where to find the exploit code. \newline
Adapt - Customize the exploit, so it fits
Not every exploit work for every system "out of the box". \newline
Try - Get ready for (lots of) trial and error. \newline


%%%%%%%%%%%%%%%%%%%%%%%%%%%%%%%%%%%%%%%%%%%%%%%%%%%%%%%%%%%%%%%%%%%%%%%%%%%%%%%%%%%%%%%%%%%

\subsection{Docker}
\label{subsubsec:docker}
Docker is a tool designed to make it easier to create, deploy, and run applications by using containers.
Containers allow a developer to package up an application with all of the parts it needs, such as libraries and other dependencies, and ship it all out as one package.
By doing so, thanks to the container, the developer can rest assured that the application will run on any other Linux machine regardless of any customized settings that machine might have that could differ from the machine used for writing and testing the code.

In a way, Docker is a bit like a virtual machine.
But unlike a virtual machine, rather than creating a whole virtual operating system, Docker allows applications to use the same Linux kernel as the system that they're running on and only requires applications be shipped with things not already running on the host computer.
This gives a significant performance boost and reduces the size of the application.

In order to install docker go to \hyperref{https://docs.docker.com/install/linux/docker-ce/ubuntu/} and follow the instructions.
Once you have docker installed you run it by \q{service docker start}.

\paragraph{Example} Install ubuntu image.
Run the command \q{docker run ubuntu}.
Now the docker checks if the ubuntu image is on your local machine (you can check it by \q{docker ps -a}).
If is not present docker pulls it from docker repository (hub) and deploys it on your machine.
If you run \q{docker run -it ubuntu} it will start new container and you will be able to interact with shell.
To remove container just type \q{docker container rm <container\_id> }


For example docker allows you to install an image of any "docked" application.
For instance