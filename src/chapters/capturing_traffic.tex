\section{Capturing traffic}
\label{sec:traffic}

%TODO add more details
\subsection{BurbSuite}
\label{subsec:burp}

Burb suite allows you to intercept (capture) information from/to websites through a proxy.
It basically captures only HTTP/S requests going through your client.
When configuring Burb suite make sure that Burb suite works on localhost and port 8080 (obviously you can choose another one but must be the same as for browser).
Set manually your browser proxy on local host to 8080.
In case when another application is listening on port 8080 change it to different port.
It means: always you want to send a request from the client to website it passes through the proxy on port 8080 and is passed to the Burb suite.
Burb suite can manipulate requests and forward them after changes.
It allows you ot manipulate requests and their parameters.
When you set up proxy settings turn Burb suite and your browser.
From now all requests you send through browser are catched in burb and you can change their parameters and forward them farther.
\paragraph{Hints} shortcuts
\newline
\q{Ctrl+T  - toggle proxy interception}
\newline
\q{Ctrl+Shift+P - switch to proxy}
\newline
\q{Ctrl+Shift+I - switch to intruder}
\newline
\q{Ctrl+Shift+R - switch to repeater}
\newline
\q{Ctrl+Minus - previous tab}
\newline
\q{Ctrl+Equals - next tab}
\newline
\q{Ctrl+R - send to repeater}
\newline
\q{Ctrl+I - send to intruder}
\newline
\q{Ctrl+F - forward intercepted message}
\newline
\q{Ctrl+U - URL encode characters}
\newline
\q{Ctrl+H - HTML encode characters}
\newline
\q{Ctrl+B - Base-64 encode}




\subsubsection{Target module}
Target module consist of Site map and Scope.
Site map lists all the pages and sublists all the subpages that you visited (or your client has redirected to in some way).
You can add them here to Scope which is the area of your interests.
Choose the target pages and add them to scope (right click/add to scope).
From now you will not be working outside the target pages so your penetration test will be safe (you will not break out-of-scope pages).

\subsubsection{Proxy module}
Most important parts of Proxy module are Interception and HTTP history.
If you are on interception tab and interception it turned on, all HTTP requests are stopped here.
You can modify each request body or header.
Changing bodies of HTTP requests is very powerful because you can check for instance if server validates input data in the same way as front end parts do.
You can forward the intercepted request to server (or another proxy) or drop it.
After forwarding the request you will get the response that you can easily analise.
In options you can choose how interception is being done, you can set up any needed rules for intercepting responses and requests.

\subsubsection{Spider module}
This module works as crawler, but not entirely - it does not use any word list in order to investigate the web page.
It simply clicks on everything on web page.
It might be dangerous because by clicking on everything you can remove something or delete user etc, basically do some unessesary stuff.
In order to use spider go to target site map, right click on website you want to crawl through and (... you should know what to do next).
Afterwards all the spidered pages will be shown in target module.


\subsubsection{Scanner module}

Scanner module is vulnerability scanner and is present only in professional (non-free) version.

\subsubsection{Intruder module}

%todo add reference to hydra and subsec:bruteforce
In some cases you might want to use repeater for request multiplication.
Considering you have input field in the application and the does not count number of requests made on it.
In other words if you can call the endpoint multiple times without blocking the IP address you can use intruder mode.
There are also other tools helping to conduct such scenario, for more info see brute force attacks.\newline \newline
In order to use intruder you must (like always) right click on the request from request list and send it to intruder.
You can (like in repeater) add each request into new fresh tab which makes your work clean.
Usually intruder module will preselect body variables to help you, but you can choose your own positions.
There are four tabs in intruder mode: target, positions,payload and options.
In target tab you configure the target host and ports.
In positions tab there are preselected positions of request variables within the body.
If you do not accept preselection you can use buttons and add, clear, or set auto selection.
Also there are four attack types to select from the list: Sniper, Battering ram, Pitchfork, Cluster bomb.

\paragraph{Sniper}
This uses a single set of payloads.
It targets each payload position in turn, and places each payload into that position in turn.
Positions that are not targeted for a given request are not affected - the position markers are removed and any enclosed text that appears between them in the template remains unchanged.
This attack type is useful for fuzzing a number of request parameters individually for common vulnerabilities.
The total number of requests generated in the attack is the product of the number of positions and the number of payloads in the payload set.
In other words if we have 5 inputs and 10 payload elements intruder will run 50 times.
\paragraph{Battering ram}
This uses a single set of payloads.
It iterates through the payloads, and places the same payload into all of the defined payload positions at once.
This attack type is useful where an attack requires the same input to be inserted in multiple places within the request (e.g. a username within a Cookie and a body parameter).
The total number of requests generated in the attack is the number of payloads in the payload set.
In other words if we have 5 inputs and 10 payload elements intruder will run 10 times.
\paragraph{Pitchfork}
This uses multiple payload sets.
There is a different payload set for each defined position (up to a maximum of 20).
The attack iterates through all payload sets simultaneously, and places one payload into each defined position.
In other words, the first request will place the first payload from payload set 1 into position 1 and the first payload from payload set 2 into position 2;
the second request will place the second payload from payload set 1 into position 1 and the second payload from payload set 2 into position 2, etc.
This attack type is useful where an attack requires different but related input to be inserted in multiple places within the request (e.g. a username in one parameter, and a known ID number corresponding to that username in another parameter).
The total number of requests generated in the attack is the number of payloads in the smallest payload set
In oder words if you have 5 inputs you choose and 5  payload elements intruder will run one time, if you have 5 inputs and 10 payload elements it will run 2 times etc.
\paragraph{Cluster bomb}
This uses multiple payload sets.
There is a different payload set for each defined position (up to a maximum of 20).
The attack iterates through each payload set in turn, so that all permutations of payload combinations are tested.
If there are two payload positions, the attack will place the first payload from payload set 2 into position 2, and iterate through all the payloads in payload set 1 in position 1;
it will then place the second payload from payload set 2 into position 2, and iterate through all the payloads in payload set 1 in position 1.
In other words this module will run all possible permutations of payloads and inputs.
This attack type is useful where an attack requires different and unrelated or unknown input to be inserted in multiple places within the request (e.g. when guessing credentials, a username in one parameter, and a password in another parameter).
The total number of requests generated in the attack is the product of the number of payloads in all defined payload sets - this may be extremely large.
\newline
Once you have chosen the attack type you will see different payload options in payload tab.
You can also configure them here.
You can provide your own word list by copying it and pasting in PayloadOptions after choosing PayloadType to Simple list.
Afterwards just go to "Intruder" tab in the main menu bar and start attack.
After your attack is finished you can compare chosen responses in comparer (see \ref{subsubsec:comparer}).

\subsubsection{Comparer}
\label{subsubsec:comparer}
It compares selected requests or responses.
Basically click on few (at least two) requests and send it (or their responses) to comparer.
Inside comparer you can compare then from byte or word point of view.

\subsubsection{Repeater module}

Repeater allows you basically to resend intercepted requests as many times as you want.
In order to send request to repeater right click on request (it does not matter wher it is) and send it to repeater.
Now in the repeater tabno can read responses modify them before reach the client, modify requests before reach server etc.
Simple is that.


\subsubsection{Sequencer module}
Sequencer is more sophisticated but still easy to use module that checks the entrophy or randomness of given tokens, cookies or other weird parameters.
Like always send chosen request to the sequencer.
Usually if your request consist of one token, cookie or something you want to check it will appear automatically in the Token Location Within Response section.
You can also customize it by using regexps in Configure tab.
Once you clicked on Configure button you can type boundary expressions in input fields or just select them within request headers and it will be automatically added to regexp section.
If you are sure that you have chosen proper points to measure the entropy click on Start Live Capture.
Simple is that.
Usually it takes 20 000 requests to measure the randomness but you can stop it and analyse also smaller amount of data.
During the measurement you can copy tokens to the text file so it will help you to analyse then in case entrophy is very low.

%todo add analyse + graph pictures
\subsubsection{Other modules}

\subsubsection{Burp collaborator}
\label{subsubsec:burp_collab}
Burp Collaborator is a network service that Burp Suite uses to help discover many kinds of vulnerabilities.
For example: Some injection-based vulnerabilities can be detected using payloads that trigger an interaction with an external system when successful injection occurs.
For example, some blind SQL injection vulnerabilities cannot be made to cause any difference in the content or timing of the application's responses, but they can be detected using payloads that cause an external interaction when injected into a SQL query.
\p{scale=0.3}{../figures/download.png}{Burp collaborator working schema}{Burp collaborator working schema}{burp_collaborator_schema}


adsfasdf

\subsection{Wireshark}
\label{subsec:wireshark}

Wireshark is the tool that connects to particular interface(s) in order to capture traffic.
It allows you to capture traffic on your local ethernet or interface as well as on router.
Wireshark is very powerful tool in networking area and some of its features are described below.

\subsubsection{Capturing traffic on Ethernet interface}

After running Wireshark you will be prompted to choose an interface to work with.
If you see no interfaces you must run Wireshark in administrator mode.
Once you have chosen Ethernet interface you will be moved to to traffic window.
Click on "start capturing" in order to record traffic on Ethernet interface.
Currently you capture all the packages going through chosen interface.
And here all means ALL.
You are basically capturing everything that goes through your network interface.
You are capturing packages and protocols from all OSI layers (Check appendix for more detailed description of layers) of data transmission: \newline \newline
Layer 7 - Application \newline
Layer 6 - Presentation \newline
Layer 5 - Session \newline
Layer 4 - Transport \newline
Layer 3 - Network \newline
Layer 2 - Data Link \newline
Layer 1 - Physical \newline


You will be overloaded by amount of data in capturing window.
In order to sort it in some usefull way special filters can be used.
You can type basically \texttt{http} or \texttt{ssl} in filter pane and confirm it (you might need to restart capturing).
Afterwards you will see only http or tls protocols data in capturing window.
You can obviously set up other filters if needed, they can be related to variety of protocols.
You can click on one line and in window below you will see data from the given protocol as well as from protocols included in the request.
In order to customize filters click on "Expression..." which is located by the filter pane.
