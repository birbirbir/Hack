
\section{Investigations and information gathering}
The most important thing in software testing are is to gather all possible and needed information about targets.
The more information you are able to analyze, the better and the more valuable test you are able to perform.
Obviously you can proceed with exploration testing with no knowledge about target but it only means you will gather this knowledge
during the investigation what is the point of exploration testing.
Remember that during information gathering phase the most important thing is to keep all info sorted in one place.
You can install program called CherryTree for information storage purposes.


% -------------------------- NETWORK SCANNING --------------------------------------
%%%%%%%%%%%%%%%%%%%%%%%%%%%%%%%%%%%%%%%%%%%%%%%%%%%%%%%%%%%%%%%%%%%%%%%%%%%%%%%%%%%%%%%
\subsection{Mapping network tools}
%\texttt{} : \texttt{ }

\subsubsection{\texttt{whois <options> domain name}}
This tools checks domain registration details like owner, technical contact, OS etc.

\subsubsection{\texttt{nmap <options> ip/ip range <options> <>}}
This if powerfull tool allowing to scan ports or other system information on one or more then one IPs.
\newline
\newline

\texttt{<XXX.XXX.XXX.XXX-XXX>} - Example IP range.
\newline

\texttt{-A} - Shows system parameters.
\newline

\texttt{-oA} - Output in three major formats at once (.nmap, .gnmap (grepable), XML) e.g.: \texttt{-oA <filename>}
\newline

\texttt{-p} - Port range e.g: \texttt{-p 0-500}
\newline

\texttt{-sS} - SYN scan, "half-open scan", is part of TCP scan but without ACK (TCP handshake consist of SYN/SYN-ACK/ACK).
SYN checks if ports are available but does not start the connection.: \texttt{-p 0-500}.
\newline

\texttt{-sV} - Version scan, gives more detailed information about target.
\newline

\texttt{-sU} - UDP scan (User Datagram Protocol is a part of core members of Internet Protocol Suite).
Nmap here sends a UDP packet to a port.
If port is not open nmap recives "port unreachable" message.
Usually this scan shows different ports than other scans and it takes more time to proceed.
\newline



\subsubsection{\texttt{NetCat: nc <options> hostname port(s)}}
Netcat, usually abbreviated to nc, is a network utility with which you can use TCP/IP protocols to read and write data across network connections.
You can use it to create any kind of connection as well as to explore and debug networks using tunneling mode, port-scanning, etc.
Checks whether given ports are listening.

\texttt{-v} - Verbose, use twice (-vv) to get more info, e.g: \texttt{nc -v 127.0.0.1  8080}

\subsubsection{\texttt{nslookup}}
DNS translator. Example : \texttt{nslookup www.google.com}

%TODO add more details
\subsubsection{\texttt{host}}
Ask for a servers for particular domain.

\subsubsection{\texttt{theHarvester}}
Phyton tool quickly goes through many searching engines for possible email addresses.
\newline

\texttt{-d} - Domain name
\newline

\texttt{-l} - Limit outcomes
\newline

\texttt{-b} - Specify search engines e.g.: \texttt{ theharvester -d cogniance.com -l 100 -b all }


%TODO add more details
\subsection{\texttt{Maltego}}
\label{subsec:maltego}
This data-mining tool visualises open source intelligence gathering.
In order to run maltego type in command line \texttt{maltego}.
You will be asked to register new account for free usage (maltego has also paid version which is more powerfull).



\subsubsection*{\texttt{}}

%%%%%%%%%%%%%%%%%%%%%%%%%%%%%%%%%%%%%%%%%%%%%%%%%%%%%%%%%%%%%%%%%%%%%%%%%%%%%%%%%%%%%%%
\subsection{Hunting for vulnerabilities, NESSUS}

Nessus has already been installed on Kali VM. In order to run Nessus you must run daemon by typing: \texttt{service nessusd start} in the terminal.
Nessus by default is running on \texttt{https://kali:8834 port}.
If you want to enter Nessus from outside of Kali you must type \texttt{https://<kali\_ip\_address>:8834}.

\subsubsection{Policies}

Policies are the set of rules explaining what to do during scan or how to behave when vulnerability has been found.
In order to create new policy go to Policies/New Policy.

\subsubsection{Scan}

In order to prepare new scan go to My Scans/New Scan.
Set up all needed information and set target(s) IP(s).
After scanning all data about vulnerabilities can be found in my Scans section.


%%%%%%%%%%%%%%%%%%%%%%%%%%%%%%%%%%%%%%%%%%%%%%%%%%%%%%%%%%%%%%%%%%%%%%%%%%%%%%%%%%%%%%%

\subsection{Web application scanning. Web crawlers}
\label{subsec:webscanning}

In general web application scanners work from two points of view.
First is when we know all the endpoints of the web application and we put that endpoints as input for scanner.
Then scanner "somehow" is looking for application vulnerabilities like hidden fields, comments hardcoded values etc.
Second method is when we do not know endpoints of the application so we need to discover them.
The process of endpoints discovery is called "crawling".
Web crawlers are forced with word lists and use them as assumption of endpoint names by sending thousands of (usually HTTP GET) requests on prepared endpoints.
For example if word list consist of one word "hackme" and host name is 10.10.10.10 crawler will be looking for \texttt{10.10.10.10/hackme} endpoint.
The same if word list consist of many words (millions or billions) discovered endpoint list grows as tree and iv time consuming.
The positive value is that crawlers give you a lot of information about website structure and give you many ideas which endpoints could be vulnerable.
\subsubsection{nikto}
Nikto is web application vulnerability scanner.
Nikto output verboses information and potential vulnerabilities in terminal.
This information are readable suggestions.
\newline
\texttt{nikto -h google.pl}

\subsubsection{dirb}

Dirb scanner uses predefined wordlist in order to check all child directories for provided parent one node eg:
\newline
\texttt*{dirb 'http://10.10.10.76/admin/api'}

\subsubsection{crawler (wmap)}
It is actually powerful metasploit module and it is called: \texttt{auxilary/crawler/msfcrawler} and its being used in the same way as other metasploit modules.

\subsubsection{websploit}
\label{subsubsec:websploit}

This is a substitute for metasploit for web applications and its usage is very similar.
One websploit module works similarly to wmap and it is called: \texttt{web/dir search}.
Usage is as regular websploit module.


\subsubsection{spider}
Spider is a module of burp suite that scans the application.
See more in \ref{subsec:burp}

\subsubsection{gobuster}
\label{subsubsec:gobuster}
Gobuster is a tool used to brute-force on URLs (directories and files) in websites and DNS subdomains.
Gobuster can be downloaded through the apt- repository.
Example syntax is: \newline
\q{gobuster -u http://192.168.1.108/pages -w /usr/share/wordlists/dirb/common.txt}


\subsubsection{amass}
\label{subsubsec:amass}
Amass is probably the most powerfull scraper written in C.
Amass looks for subdomains for particular webdomain.
Let's download it via docker.
Type in the commandline: \newline
\q{sudo docker build -t amass https://github.com/OWASP/Amass.git} \newline
Now you can run the image against the given domain by: \newline
\q{sudo docker run amass --passive -d zbirek.wow}

The Amass also needs a wordlist to work with but in this time the wordlist is included in the \q{/wordlist/} within the container.
You can run: \q{sudo docker run amass -w /wordlists/all.txt -d example.com}

