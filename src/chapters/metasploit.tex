
\section{Metasploit}
Metasploit is a framework for penetration testing, which has included the whole base of vulnerabilities for different systems and API that allows testers to exploit or scan targets.
By vulnerabilities here are meant points of the target which are not secured properly and are vulnerable for attack e.g:
login or password buffer overflow on target's database server.
Metasploit uses predefined exploits and payloads in order to get control over the target system.
Exploit is technically program (procedure) that uses shellcode scripts in order to exploit vulnerabilities Each exploit and payload must be configured for particular target.




%%%%%%%%%%%%%%%%%%%%%%%%%%%%%%%%%%%%%%%%%%%%%%%%%%%%%%%%%%%%%%%%%%%%%%%%%%%%%%%%%%%%%%%
\subsection{msfconsole}
In order to use metasploit framework you can go directly to Applications->Exploitation tools and go to Metasploit.
This approach will run the terminal simultaneously running all
data bases and services in the background.
If you wish you can run Metasploit directly from terminal by typing \texttt{msfconsole} but you might need to run postgresql service (\texttt{service postgresql start}).

\subsubsection{Searching for exploits, run exploit}

Now you need to see the info, options and payloads for the exploit by typing: \texttt{info, options, show payloads} respectively.
If you know what exploit you will be using you can directly type into msf console the following line:
\texttt{use <exploit/name>}.
If you do not know the exploit you can check it online on www.rapid7.com
or by typing in msf console \texttt{search <exploit/tag/orname>} and then you can \texttt{use} it.
Afterwards you are moved to exploit subconsole.
Now you need to set the payload: \texttt{set payload <payload/name>},
then if you type \texttt{options} again, you will see all mandatory options (for listener and reciever).
Now is time for setting up the options e.g: \texttt{set RHOST 192.168.4.56} or \texttt{set LHOST 192.168.4.56}
if you are using reverse shell (pushing the connection back to the listener (attack machine) rather than
waiting for an incoming connection from target which is more likely to make connection through firewall \cite{pen}).

\subsubsection{Outputing data}
If you want to output data somewhere (which is VERY usefull) you can use spooling before you start exploiting.
Just type in msf console (before running exploit) \texttt{spool <file-name.txt/log/etc>}.
In order to turn off spooling use \texttt{spool off}.
Spool saves the whole terminal output stream.
\paragraph{Example}
In this example you will get control over the Windows XP SP3 system which you installed as VM. In msfconsole run the following: \newline\newline
\texttt{use/windows/pop3/seattlelab\_pass} \newline
\texttt{set RHOST XXX.XXX.XXX.XXX} (your target IP) \newline and run the exploit by typing: \newline \texttt{exploit} \newline \newline

Now you are in meterpreter sub console.
Check the following commands: \texttt{help, background, pwd, shell}, for more commands go to: https://www.offensive-security.com/metasploit-unleashed/meterpreter-basics/ .
So what has happend ?
You exploited unauthenticated buffer overflow vulnerability in the POP3 server of Seattle Lab Mail 5.5 by sending a password with excessive length (for more info check exploit info in msfconsole, for more info about buffer overflow go to Buffer Overflow section in this article).

\subsubsection{msfcli}
MetaSploit Framework Command Line Interpreter is another way to interact with Metasploit.
Is especially useful while scripting Metasploit commands or while testing new payloads in automated way.


%%%%%%%%%%%%%%%%%%%%%%%%%%%%%%%%%%%%%%%%%%%%%%%%%%%%%%%%%%%%%%%%%%%%%%%%%%%%%%%%%%%%%%%
\subsection{Metasploit Scanner Modules and Check functions}

Metasploit serves vulnerability searching modules, these modules begin from \texttt{scanner/}.
In order to use metasploit scanner treat scanner as regular module and run is with \texttt{use <module\_name>}, set up options and run exploit.
Example module can be \texttt{scanner/ftp/anonymous} which searches for anonymous access via FTP ports, which can be dangerous when some important data
are stored on client side and are allowed to use FTP port.

Some exploits have embeeded \texttt{check} functions.
They can be used instead of \texttt{exploit} function in order to literally check if target is vulnerable to given exploit.


%%%%%%%%%%%%%%%%%%%%%%%%%%%%%%%%%%%%%%%%%%%%%%%%%%%%%%%%%%%%%%%%%%%%%%%%%%%%%%%%%%%%%%%
\subsection{Armitage}

Armitage is GUI for metasploit enriched by many automated tools helping a tester figure out what exploit should be used (what is the target vulnerability).
In order to run Armitage go to Applications-> Exploitation Tools.
Login to database with default credentials.
Armitage is good option if you have no ideal what exploit you should use.
There two options to get that knowledge.
First is to check all possible attacks and choose preferred one, second is run Heil Mary and perform all possible attacks one after another.
The difference is that first method
only checks for available exploits for the target whereas the second one runs all exploits without asking.
The problem is that the second one can really crash the system, so it is not recommended but still worth to work around.
First of all go to: Hosts->Nmap->Intense\_scan or Hosts->MSFscan
, you will then get your local network scanned so you will know what systems are there and what exploit you will be able to use.
In order to perform first option go to: Hosts->Find\_attacks.
After this you will need click on selected machine and choose check all possible attacks separately, then choose proper exploit and use it.
If you want to use second option go to Hosts>Heil Mary, after this all possible exploits will be ran on the target system.
If your system is vulnerable and can be affected, the system "screen" icon will changed and you will be informed about action taken over the system.
Then you can run meterpreter from target options and use the same commands like in terminal meterpreter.