
\section{Attacks}

In this section few types of attacks and related to them tools are described.

\subsection{Brute force attack}
\label{subsec:bruteforce}
This attack relates to forcing the target.
In this scenario you want to send thousands of requests on one endpoint or set of endpoints.
In such case you think that after such stack of tries you will find the solution.
There are many ways and tools to help you in this attack and they are mentioned below.

\subsubsection{Hydra}
\label{subsubsec:hydra}
Hydra is a parallelized login cracker which supports numerous protocols to attack.
It is very fast and flexible, and new modules are easy to add.
This tool makes it possible for researchers and security consultants to show how easy it would be to gain unauthorized access to a system remotely.

The basic syntax is: \texttt{hydra -L <path\_to\_login\_wordlist> -P <path\_to\_pass\_wordlist> <target\_ip> }

You can also use it for endpoints that does not need pass and login, it can be used for any kind of input.

\subsubsection{Curl}
Curl is mentioned here but it is not attack-focused tool.
It is mentioned here since it allows you send multiple requests in looped way by scripting in bash.
In order to sent POST request on \texttt{/registration} endpoint you should send the following request multiple times:\newline \newline
\texttt{curl -d "param1=value1\&param2=value2" -X POST http://<remote\_or\_local\_host>:3000/data} \newline
where param1 and param 2 can be login and password.
For more detailed info about scripting in curl see\ref{subsubsec:curl}.